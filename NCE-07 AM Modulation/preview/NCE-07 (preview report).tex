% 若编译失败,且生成 .synctex(busy) 辅助文件,可能有两个原因:
% 1. 需要插入的图片不存在:Ctrl + F 搜索 'figure' 将这些代码注释/删除掉即可
% 2. 路径/文件名含中文或空格:更改路径/文件名即可

% --------------------- 文章宏包及相关设置 --------------------- %
% >> ------------------ 文章宏包及相关设置 ------------------ << %
% 设定文章类型与编码格式
\documentclass[UTF8]{article}		
\input{../../.config/config_for_NonlinearCircuitExperiment.tex}



%%%%%%%%%%%%%%%%%%%%%%%%%%%%%%%%%%%%%%%%%%%%%%%%%%%%%%%%%%%%%%%%
% 仅需修改页眉、实验名称、实验日期
%%%%%%%%%%%%%%%%%%%%%%%%%%%%%%%%%%%%%%%%%%%%%%%%%%%%%%%%%%%%%%%%


%%%%%%%%%%%%%%%%%% 1. 修改页眉内容 %%%%%%%%%%%%%%%%%%
\rhead{Preview Report of NCE-07 AM Mod (2025.12.11, 丁毅)}

% 开始编辑文章
\begin{document}
\begin{center}\large
    \vspace*{-0.8cm}
    \noindent{\huge\bfseries《\ \ 非\ \ 线\ \ 性\ \ 电\ \ 路\ \ 实\ \ 验\ \ \ 》\ \ 预\ \ 习\ \ 报\ \ 告 }
    \\\vspace{0.1cm}
    \noindent{
    {\bfseries 
%
%%%%%%%%%%%%%%%%%% 2. 修改实验名称 %%%%%%%%%%%%%%%%%%
    实验名称:\uline{\hspace{0.6cm} Standard AM Modulation \hspace{0.6cm}}
%
    }\hspace{0.4cm}
    指导教师:\uline{\hspace{0.5cm}冯鹏\ \ fengpeng06@semi.ac.cn     \hspace{0.5cm}}
    }
    \\\vspace{0.1cm}
    \noindent
    {
    姓名:\uline{\,\,\,丁毅\,\,\,}\hspace{0.2cm}
    学号:\uline{\,\,\,{ 2023K8009908031}\,\,\,}\hspace{0.2cm}
    班级/专业:\uline{\,\,\,{2308/电子信息}\,\,\,}\hspace{0.2cm}
    分组序号:\uline{\,\,\,{2-06}\,\,\,}
    }
    \\\vspace{0.1cm}
    \noindent{
%
%%%%%%%%%%%%%%%%%% 3. 修改实验日期 %%%%%%%%%%%%%%%%%%
    实验日期:\uline{\,\,{2025.12.11}\,\,}\hspace{0.2cm}
%
    实验地点:\uline{\,\,\,西实验楼 (8 号楼) { 308}\,\,\,}\hspace{0.2cm}
    是否调课/补课:\uline{\hspace{0.1cm}否 \hspace{0.1cm}}\hspace{0.2cm}
    成绩:\uline{\hspace{0.6cm}}
    }
\end{center}
\vspace{-0.4cm}
\noindent\rule{\textwidth}{0.075em}   % 分割线
\vspace{-1.0cm}


% ------------------------ 文章信息区 ------------------------ %
% ------------------------ 文章信息区 ------------------------ %



%%%%%%%%%%%%%%%%%%%%%%%%%%%%%%%%%%%%%%%%%%%%%%%%%%%%%%%%%%%%%%%%%%%%%%%%%%%%%%%%%
%%%%%%%%%%%%%%%%%%%%%%%%%%%%%%%%% 下面是正文内容 %%%%%%%%%%%%%%%%%%%%%%%%%%%%%%%%%
%%%%%%%%%%%%%%%%%%%%%%%%%%%%%%%%%%%%%%%%%%%%%%%%%%%%%%%%%%%%%%%%%%%%%%%%%%%%%%%%%

\section{实验目的}
\begin{enumerate}
    \item 掌握基于集成模拟乘法器的幅度调制 (AM) 原理与实现方法。
    \item 理解AM信号频谱结构、功率分配及调制系数 (Modulation Index) 的影响。
    \item 掌握使用MC1496四象限模拟乘法器实现AM调制的电路配置与调试方法。
    \item 学会在示波器上测量调制系数并分析基带信号 (原始信号)、载波与已调波之间的关系。
\end{enumerate}

\section{实验仪器}

\begin{enumerate}
\item 高频实验箱 - 乘法调幅/混频实验板 (031132201809392)
\item 示波器 RIGOL MSO2202A  (080103201901376)
\item 信号发生器 GWINSTEK AFG-2225  (080102201901355)
\item 万用表 LINIT- UT61A (C181503983)
\end{enumerate}


\section{实验原理}
\subsection{幅度调制 (Amplitude Modulation) 原理}

\noindent 设基带信号 (原始信号) 和载波 (carrier) 信号分别为:
\begin{gather}
v_s(t) = A_s \cos(\omega_s t),\quad v_c(t) = A_c \cos(\omega_c t)
\end{gather}
则 Standard (Classic) Amplitude-Modulated Signal (标准幅度调制后的信号,后文简称“调制信号”) 可表示为:
\begin{gather}
v_{AM}(t) = A_c \big[ 1 + m \cos(\omega_s t) \big] \cos(\omega_c t),\quad m = \frac{A_s}{A_c}
\end{gather}
其中 $m \in (0,\ 1)$ 为幅度调制系数 (std. AM modulation factor),一般不超过 1,否则会发生过调制 (Overmodulation) 从而引起失真。



\subsection{Amplitude-Modulated Signal 频谱与功率分析}
\noindent 将上面标准幅度调整后的信号  $v_{AM}(t)$  展开可得:
\begin{gather}
v_{AM}(t) = A_c \cos(\omega_c t) + \frac{mA_c}{2} \cos[(\omega_c + \omega_s)t] + \frac{mA_c}{2} \cos[(\omega_c - \omega_s)t]
\end{gather}
假设原始信号为理想单频正弦波,此时 $v_{AM}(t)$ 的频谱有且仅有三个频率分量:
\begin{gather}
\text{Carrier:\ \ } f_c
,\quad 
\text{USB (Upper Side Band):\ \ } f_c + f_s
,\quad
\text{LSB (Lower Side Band):\ \ } f_c - f_s
\\
P_{\text{total}} = P_c + P_{\text{USB}} + P_{\text{LSB}} = P_c \left( 1 + \frac{m^2}{2} \right),\quad P_{\text{USB}} = P_{\text{LSB}} = \frac{m^2}{4} P_c
\end{gather}

边带功率随 $m$ 的增大而增加,当 $m = 1$ 时取到最大边带功率 $P_{\text{USB}} = P_{\text{LSB}} = \frac{1}{4} P_c$,此时上下边带功率分别为载波功率的四分之一,意味着大多数功率都被不含信息的载波占据,浪费严重,这也是标准幅度调制效率低下的主要原因。但是由于这种调制和解调设备简单,方便实现,所以仍有不少应用场景。


\subsection{MC1496 模拟乘法器工作原理}
MC1496 为经典的双平衡四象限模拟乘法器 (Gilbert Cell 结构),内部由三组差分对构成,可实现两路输入信号的线性相乘:
\begin{gather}
v_{\text{out}} = k \cdot v_x \cdot v_y
\end{gather}

在AM调制中,将基带信号 (原始信号)  $v_s(t)$  与直流偏置 $V_0$ 叠加后输入X 端,载波  $v_c(t)$  输入 Y 端,即可在输出端得到 AM Signal:
\begin{gather}
v_x(t) = v_s(t) + V_0 = A_s \cos(\omega_s t) + V_0,\quad 
v_y(t) = v_c(t) = A_c \cos(\omega_c t)
\\
v_{out}(t) = k A_c V_0 \left[ 1 + \frac{A_s}{V_0} \cos(\omega_s t) \right] \cos(\omega_c t) = A_c' \left[ 1 + m \cos(\omega_s t) \right] \cos(\omega_c t) 
\\ 
\text{where:\ \ }
A_c' = k A_c V_0,\quad m = \frac{A_s}{V_0}
\end{gather}
这恰好符合标准幅度调制信号的形式,也是标准幅度调整最常用的实现方法。

\subsection{实验电路简要分析}
\noindent 如 Figure \ref{fig__circuit} 所示,实验电路基于 MC1496 构成,外围电路包括:
\begin{enumerate}
    \item 偏置电阻网络:设置静态工作点
    \item 可调电位器RW1、RW2、RW3:分别用于调节输出波形对称性、调制系数和输出幅度
    \item Emitter Follower (Common Collector) 输出级:提高输出驱动能力,隔离负载对调制电路的影响
\end{enumerate}

\begin{figure}[H]\centering
    \includegraphics[width=\columnwidth]{assets/circuit.png}
    \caption{MC1496-Based AM Modulation Circuit Used in the Experiment}
    \label{fig__circuit}
\end{figure}

\section{实验内容与步骤}

\subsection{实验前跳线设置}
\begin{enumerate}
    \item J1、J3、J5:置于 1-2 位置 (调制功能) 
    \item J2、J8、J9:置于 2-3 位置 (调制功能) 
    \item IN1:空闲
    \item IN2:输入基带信号 (原始信号) (1 kHz sine wave @ 300 mVpp) 
    \item IN3:输入载波信号 (10.7 MHz sine wave @ 500 mVpp)
\end{enumerate}

\subsection{调节步骤}
\begin{enumerate}
    \item 先仅在 IN3 输入载波信号 ,调节 RW1/RW2/RW3 使 OUT 端输出不失真正弦波;
    \item 然后接入基带信号 (原始信号),微调 RW1 (对称性) 、RW2 (调制系数 $m$ ) 、RW3 (输出幅度),使输出为典型 AM 波形;
    \item 使用示波器对输出波形进行采样,测量最大幅度 $A_{\max}$ 和最小幅度 $A_{\min}$ 并计算调制系数 $m = \frac{A_{\max} - A_{\min}}{A_{\max} + A_{\min}}$。
    \item 调节 $m$ 分别为 0.2, 0.4 和 0.7,记录对应 $A_{\max}$、$A_{\min}$ 值及输出波形。
\end{enumerate}



\section{思考题}

\subsection{如何利用频谱分析仪测得普通调幅波的调制系数?}

在频谱仪上可测得载波幅度 $A_c$ 与两个边带幅度 $A_{\text{USB}}$、$A_{\text{LSB}}$,由此计算调制系数 $m$:
\begin{gather}
m = \sqrt{\frac{(A_{\text{USB}}^2 + A_{\text{LSB}}^2)}{A_c^2}} = \frac{\sqrt{(A_{\text{USB}}^2 + A_{\text{LSB}}^2)}}{A_c}
\end{gather}








\end{document}

% VScode 常用快捷键:

% F2:                       变量重命名
% Ctrl + Enter:             行中换行
% Alt + up/down:            上下移行
% 鼠标中键 + 移动:           快速多光标
% Shift + Alt + up/down:    上下复制
% Ctrl + left/right:        左右跳单词
% Ctrl + Backspace/Delete:  左右删单词    
% Shift + Delete:           删除此行
% Ctrl + J:                 打开 VScode 下栏(输出栏)
% Ctrl + B:                 打开 VScode 左栏(目录栏)
% Ctrl + `:                 打开 VScode 终端栏
% Ctrl + 0:                 定位文件
% Ctrl + Tab:               切换已打开的文件(切标签)
% Ctrl + Shift + P:         打开全局命令(设置)

% Latex 常用快捷键:

% Ctrl + Alt + J:           由代码定位到PDF


