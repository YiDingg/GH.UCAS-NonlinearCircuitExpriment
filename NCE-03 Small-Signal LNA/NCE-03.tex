% 若编译失败,且生成 .synctex(busy) 辅助文件,可能有两个原因:
% 1. 需要插入的图片不存在:Ctrl + F 搜索 'figure' 将这些代码注释/删除掉即可
% 2. 路径/文件名含中文或空格:更改路径/文件名即可

% --------------------- 文章宏包及相关设置 --------------------- %
% >> ------------------ 文章宏包及相关设置 ------------------ << %
% 设定文章类型与编码格式
\documentclass[UTF8]{article}		
\input{../.config/config_for_NonlinearCircuitExperiment.tex}



%%%%%%%%%%%%%%%%%%%%%%%%%%%%%%%%%%%%%%%%%%%%%%%%%%%%%%%%%%%%%%%%
% 仅需修改页眉、实验名称、实验日期
%%%%%%%%%%%%%%%%%%%%%%%%%%%%%%%%%%%%%%%%%%%%%%%%%%%%%%%%%%%%%%%%


%%%%%%%%%%%%%%%%%% 1. 修改页眉内容 %%%%%%%%%%%%%%%%%%
\rhead{NCE-03 Resonant Amplifier (2025.11.27, 丁毅)}

% 开始编辑文章
\begin{document}
\begin{center}\large
    \vspace*{-0.8cm}
    \noindent{\huge\bfseries《\ \ 非\ \ 线\ \ 性\ \ 电\ \ 路\ \ 实\ \ 验\ \ \ 》\ \ 实\ \ 验\ \ 报\ \ 告 }
    \\\vspace{0.1cm}
    \noindent{
    {\bfseries 
%
%%%%%%%%%%%%%%%%%% 2. 修改实验名称 %%%%%%%%%%%%%%%%%%
    实验名称:\uline{\hspace{0.2cm} Small-Signal Resonant Amplifier  \hspace{0.2cm}}
%
    }\hspace{0.4cm}
    指导教师:\uline{\hspace{0.2cm}冯鹏\ \ fengpeng06@semi.ac.cn     \hspace{0.2cm}}
    }
    \\\vspace{0.1cm}
    \noindent
    {
    姓名:\uline{\,\,\,丁毅\,\,\,}\hspace{0.2cm}
    学号:\uline{\,\,\,{ 2023K8009908031}\,\,\,}\hspace{0.2cm}
    班级/专业:\uline{\,\,\,{2308/电子信息}\,\,\,}\hspace{0.2cm}
    分组序号:\uline{\,\,\,{2-06}\,\,\,}
    }
    \\\vspace{0.1cm}
    \noindent{
%
%%%%%%%%%%%%%%%%%% 3. 修改实验日期 %%%%%%%%%%%%%%%%%%
    实验日期:\uline{\,\,{2025.11.27}\,\,}\hspace{0.2cm}
%
    实验地点:\uline{\,\,\,西实验楼 (8 号楼) { 308}\,\,\,}\hspace{0.2cm}
    是否调课/补课:\uline{\hspace{0.1cm}否 \hspace{0.1cm}}\hspace{0.2cm}
    成绩:\uline{\hspace{0.6cm}}
    }
\end{center}
\vspace{-0.4cm}
\noindent\rule{\textwidth}{0.075em}   % 分割线
\vspace{-1.0cm}

% 生成目录
\setcounter{tocdepth}{3}  % 目录深度为 2(不显示 subsubsection)
\noindent\tableofcontents\thispagestyle{fancy}   % 显示页码、页眉等
%\newpage
\vspace{0.5cm}
\noindent\rule{\textwidth}{0.075em}   % 分割线

% ------------------------ 文章信息区 ------------------------ %
% ------------------------ 文章信息区 ------------------------ %



%%%%%%%%%%%%%%%%%%%%%%%%%%%%%%%%%%%%%%%%%%%%%%%%%%%%%%%%%%%%%%%%%%%%%%%%%%%%%%%%%
%%%%%%%%%%%%%%%%%%%%%%%%%%%%%%%%% 下面是正文内容 %%%%%%%%%%%%%%%%%%%%%%%%%%%%%%%%%
%%%%%%%%%%%%%%%%%%%%%%%%%%%%%%%%%%%%%%%%%%%%%%%%%%%%%%%%%%%%%%%%%%%%%%%%%%%%%%%%%


\section{实验目的}



\begin{enumerate}
\item 掌握小信号谐振放大器的工作原理;
\item 掌握小信号谐振放大器的调试方法;
\item 掌握小信号谐振放大器各项技术参数测试 (电压增益、通频带、矩形系数)。
\end{enumerate}



\section{实验仪器}

\begin{enumerate}
\item 高频实验箱 - 小信号谐振放大器实验板 (031132201809392)
\item 示波器 RIGOL MSO2202A  (080103201901376)
\item 信号发生器 GW INSTEK AFG-2225  (080102201901355)
\item 万用表 LINIT- UT61A (C181848639)
\end{enumerate}


\section{实验原理}

\subsection{小信号调谐放大器基本原理}

小信号调谐放大器 (Small-Signal Resonant Amplifier) 的作用是有选择地对某一频率范围的高频小信号进行放大,这里的“小信号”通常指电压幅度 (amplitude) 在 uV $\sim$ mV 数量级的输入信号。 \textbf{本次实验所讨论的“小信号谐振放大器”,其实类似射频系统中的 LNA (Low Noise Amplifier,低噪声放大器),} 只是工作频率 (约 10 MHz) 远低于常见无线通信频段 (例如 2.4 GHz)。 LNA 一般用于无线接收系统的前端,主要任务是对天线接收到的微弱射频信号进行放大,在保持可介绍信噪比 (SNR) 的情况下,将信号放大至足够大的功率给后续的混频器或解调器。





\begin{figure}[H]\centering
    \includegraphics[width=\columnwidth]{NCE-03 Small-Signal LNA/assets/RA schematic.png}
    \caption{Small-Signal Resonant Amplifier Schematic}
\end{figure}

因此,我们完全可以将本实验中的小信号调谐放大器视为一种低频版的 LNA,其工作原理和技术参数与 LNA 十分相似,但对线性度和噪声系数的要求没有 LNA 那么高。


小信号调谐放大器 (后文简称为 “Resonant Amplifier”) 的工作原理是:使用 RLC 谐振网络作为 BJT/MOS 放大电路的负载部分,谐振频率既为主要工作频率,使电路对带内信号 (in-band signal) 进行放大,同时有效抑制带外信号 (out-of-band signal),这与 LNA 的工作原理类似。

本次实验的 Resonant Amplifier 由两级构成,根据噪声系数的理论公式,如果想尽可能降低噪声系数,就必须保持前级的噪声系数不太高。因此,在设计前级放大器时,常要求采用低噪声器件,合理地设置工作电流等,使放大器在尽可能高的功率增益下噪声系数最小。

对实验所用电路进行简单分析:
\begin{enumerate}
\item 这是一个二级放大器,从 IN1 (TP1) 端口输入小信号,由 C3 交流耦合到第一级放大器 Q1 的 Base;
\item 第一级:由 R2/W1/R3/R4/R6 提供直流偏置 (C4 为旁路电容),C7/C6/T1 构成谐振负载网络 (C7 可调),输出通过 T1 耦合到第二级放大器 Q2 的 Base,可通过 TP2 测量第一级输出信号;
\item 第二级:由 R7/W2/R9/R8/R10 提供直流偏置 (C10 为旁路电容),C12/C11/T2 构成谐振负载网络 (C12 可调),输出通过 T2 耦合到输出端口 OUT (TP3)。
\end{enumerate}


\subsection{小信号调谐放大器技术参数}

R调谐放大器的主要技术参数包括增益 (Voltage Gain)、带宽 (Bandwidth) 和矩形系数 (Shape Factor) 等:
\begin{align}
\mathrm{Voltage gain:\ \ } 
A_v &:= \frac{v_{out}}{v_{in}} \\
\mathrm{Bandwidth:\ \ } 
\mathrm{BW}_{-3 \mathrm{dB}} &= f_{H,-3 \mathrm{dB}} - f_{L,-3 \mathrm{dB}} \\
\mathrm{Shape\ Factor:\ \ } 
\mathrm{SF}_{-20 \mathrm{dB}} &:= \frac{\mathrm{BW}_{-20 \mathrm{dB}}}{\mathrm{BW}_{-3 \mathrm{dB}}}
\end{align}

上式中 $v_{in}$ 和 $v_{out}$ 分别为放大器的输入和输出电压 (电压带有幅度和相位,因此增益是复数),$\mathrm{BW}_{-3 \mathrm{dB}}$ 和 $\mathrm{BW}_{-20 \mathrm{dB}}$ 分别为放大器 voltage gain 下降 3 dB (最大值的 $\frac{\sqrt{2}}{2}$) 和 20 dB (最大值的 $\frac{1}{10}$) 时对应的频率范围宽度。注意 $\mathrm{BW} = \frac{f_0}{Q}$,因此放大器的带宽和矩形系数与谐振电路的品质因数 $Q$ 密切相关。







\section{实验内容与步骤}

本次实验的注意内容为“放大器的频率特性及通频带的测量”。需注意:
\begin{enumerate}
\item 调整两级放大器的可调电容,使两级放大器的谐振频率分别略低于和略高于 10 MHz, 但不能完全相等或相差过大;完全相等会导致频率响应曲线过于尖锐,难以测量通频带,差距过大则会导致增益明显下降,甚至出现中间窄两边高的“双峰现象”;
\item 为电路注入小信号时,输入信号幅度 (amplitude) 应控制在 30 mV 以内,过大会导致放大器进入非线性工作区,输出波形严重失真;
\end{enumerate}

\subsection{调整电路直流工作点}

\begin{enumerate}
\item 连接好电路,为电路提供直流电源后,从 IN1 端输入 $f_0 = 10 \ \mathrm{MHz}$ 小信号,幅度在 10 mV 左右;
\item 在 OUT 端用示波器观测到放大后的输入信号,调整两个电位器 W1/W2 来改变两个晶体管的静态工作点。微调电容 C7/C12,使输出信号幅度最大且失真最小。
\end{enumerate}

\subsection{放大器的放大倍数及通频带的测试}

\begin{enumerate}
\item 调整输入跳线 J1, 使电路输入端为 IN (TP1) 而不是天线;
\item 测量电压增益:分别在 TP1 (IN) 和 TP3 (OUT) 处用示波器测量输入输出信号的幅度,计算电压增益;
\item 测量带宽:在中心频率 $f_0$ 附近改变输入频率,例如 8 MHz $\sim$ 12 MHz @ 0.1 MHz step,重复步骤 (2),对结果进行拟合后计算 -3dB 带宽;
\item 测量矩形系数:根据步骤 (3) 得到的结果,计算 -20dB 带宽,进而计算矩形系数。
\end{enumerate}

~

\section{实验结果与分析}


先设置信号发生器输出幅度为 $v_{sig,amp} = 25 \ \mathrm{mV}$ (50 mVpp) @ 10 MHz,用示波器监测电路的输入输出情况。调整可变电阻和电容,直到电路具有合适的直流工作点,效果如 Figure \ref{fig: io waveforms} 所示。


注意 Figure \ref{fig: io waveforms} 中,通道一测量的 "Input Signal" $v_{in}$ 是电路输入端实际接收到的信号,与信号发生器直接输出的 $v_{sig}$ 不同。具体而言,假设电路的输入阻抗为 $Z_{in}$,而信号发生器输出阻抗 $Z_{s} = R_s = 50\ \Omega$,我们有:

\begin{gather}
v_{in} = \frac{Z_{in}}{Z_{in} + Z_s} \times v_{sig} = \frac{Z_{in}}{Z_{in} + 50\ \Omega} \times v_{sig}
\end{gather}




将示波器两通道设置为“交流耦合”,对不同输入频率下的输入输出电压进行采样。\textbf{由于示波器上直接测量得到的 peak-to-peak value 不够准确,我们将示波器采样数据导出到 MATLAB, 从频谱角度计算输入输出电压幅度。} 具体而言,对于每次测量,将采样数据导出到 MATLAB 并作 DFT (Discrete Fourier Transform) 得到信号频谱,然后再由频谱计算出电压幅度,如 Figure \ref{fig: exported waveforms}, Figure \ref{fig: input spectrum} 和 Figure \ref{fig: output spectrum} 所示。


\begin{figure}[H]\centering
    \includegraphics[width=\columnwidth]{NCE-03 Small-Signal LNA/assets/RIGOL Print Screen2025-11-27 17_23_15.575__NCE-03__IN_vs_OUT_.png}
    \caption{Input and Output Waveforms of The Resonant Amplifier @ Signal Amplitude $v_{sig,amp}$ = 25 mV (50 mVpp). CH1 (Blue): Input Voltage $v_{in}$ @ TP1; CH2 (Red): Output Voltage $v_{out}$ @ TP3.}
    \label{fig: io waveforms}
\end{figure}


计算电压幅度时需注意:不能简单地将幅度最大的单个点作为信号幅度,因为频谱分辨率 $f_r = \Delta f$ 不一定与信号频率 $f_0$ 匹配,当信号频率 $f_0$ 不是频谱分辨率 $ \Delta f$ 的整数倍时,会出现“频谱泄露”现象。此时,需要将信号频率附近的区间考虑进去,在窄区间内对功率进行积分,再折算为电压幅度。具体代码详见 “附录B MATLAB Codes”。



\newpage
\begin{figure}[H]\centering
    \includegraphics[width=\columnwidth]{NCE-03 Small-Signal LNA/assets/2025-11-27_15-08-50__NCE-03_WaveformGenerator_50mVpp_10MHz_IN_vs_OUT.png}
    \caption{Exported Input and Output Waveforms from The Oscilloscope to MATLAB}
    \label{fig: exported waveforms}
\end{figure}

\begin{figure}[H]\centering
    \includegraphics[width=\columnwidth]{NCE-03 Small-Signal LNA/assets/2025-11-27_15-05-41__NCE-03_WaveformGenerator_50mVpp_10MHz_IN.png}
    \caption{Calculate The Frequency Spectrum and Voltage Amplitude of The Input Voltage $v_{in}$ in MATLAB}
    \label{fig: input spectrum}
\end{figure}

\begin{figure}[H]\centering
    \includegraphics[width=\columnwidth]{NCE-03 Small-Signal LNA/assets/2025-11-27_15-06-35__NCE-03_WaveformGenerator_50mVpp_10MHz_OUT.png}
    \caption{Calculate The Frequency Spectrum and Voltage Amplitude of The Output Voltage $v_{out}$ in MATLAB}
    \label{fig: output spectrum}
\end{figure}



\noindent 测量和计算得到的输入输出记录在 Table \ref{tab: freq response} 中:
\begin{table}[H]\centering
    %\renewcommand{\arraystretch}{1.5} % 调整行间距为 1.5 倍
    %\setlength{\tabcolsep}{1.5mm} % 调整列间距
    \caption{Frequency Response of The Resonant Amplifier @ Signal Amplitude $V_{sig,amp}$ = 25 mV (50 mVpp)}
    \label{tab: freq response}
\resizebox{\linewidth}{!}{   % 设置宽度为 \linewidth 等比例缩放
\begin{tabular}{ccccccccccccccccccc}\toprule
    (信号发生器) $f_{in}$ (MHz) & 8.0 & 8.1 & 8.2 & 8.3 & 8.4 & 8.5 & 8.6 & 8.7 & 8.8 & 8.9 & 9.0 & 9.1 & 9.2 & 9.3 & 9.4 & 9.5 & 9.6 & 9.7 \\
    (AC of TP1) $V_{in,amp}$ (mV) & 18.06 & 17.10 & 17.82 & 17.80 & 16.98 & 18.44 & 17.16 & 17.87 & 17.80 & 16.87 & 17.80 & 16.16 & 16.04 & 14.54 & 11.69 & 09.25 & 06.12 & 06.70 \\
    (AC of TP3) $V_{out,amp}$ (mV) & 080.35 & 083.34 & 095.98 & 107.52 & 116.61 & 134.05 & 144.63 & 162.83 & 180.40 & 196.06 & 222.79 & 240.55 & 271.32 & 300.22 & 324.90 & 364.73 & 382.91 & 416.16 \\
    \midrule
    (信号发生器) $f_{in}$ (MHz) & 9.8 & 9.9 & 10.0 & 10.1 & 10.2 & 10.3 & 10.4 & 10.5 & 10.6 & 10.7 & 10.8 & 10.9 & 11.0 & 11.1 & 11.2 & 11.3 & 11.4 & 11.5 \\
    (AC of TP1) $V_{in,amp}$ (mV) & 08.69 & 10.06 & 12.21 & 12.36 & 13.67 & 14.20 & 13.88 & 14.79 & 13.46 & 13.27 & 12.60 & 11.62 & 12.31 & 11.53 & 12.17 & 12.25 & 11.82 & 12.68 \\
    (AC of TP3) $V_{out,amp}$ (mV) & 444.86 & 471.922 & 528.78 & 567.31 & 641.76 & 716.66 & 782.39 & 874.91 & 875.35 & 849.76 & 750.80 & 615.83 & 513.84 & 405.64 & 338.35 & 280.78 & 323.35 & 203.99 \\
    \bottomrule
\end{tabular}
}\end{table}

根据 Table \ref{tab: freq response} 数据计算并拟合放大器的频率响应曲线,如 Figure \ref{fig: freq response} 和 Figure \ref{fig: fitted freq response} 所示。然后进一步计算出放大器技术参数,包括 -3dB BW, -20dB BW, 矩形系数 SF 以及整个放大器的品质因数 $Q = \frac{f_0}{\mathrm{BW}_{-3\mathrm{dB}}}$:
\begin{gather}
A_{s,max} = A_s(f_0) = 17.5474 \ \mathrm{V/V} = 24.8843 \ \mathrm{\mathrm{dB}},\quad f_0 = 10.5683 \ \mathrm{MHz}
\\
f_{-3\mathrm{dB},L} = 10.177 \ \mathrm{MHz},\ f_{-3\mathrm{dB},H} = 10.909 \ \mathrm{MHz} \Longrightarrow \mathrm{BW}_{-3\mathrm{dB}} = 0.7314 \ \mathrm{MHz} 
\\ 
f_{-20\mathrm{dB},L} = 8.286 \ \mathrm{MHz},\ f_{-20\mathrm{dB},H} = 12.060 \ \mathrm{MHz} \Longrightarrow \mathrm{BW}_{-20\mathrm{dB}} = 3.7735 \ \mathrm{MHz}
\\ 
\mathrm{SF}_{-20 \mathrm{\mathrm{dB}}}= \frac{\mathrm{BW}_{-20\mathrm{dB}}}{\mathrm{BW}_{-3\mathrm{dB}}} = 5.1592,\quad 
Q = \frac{f_0}{\mathrm{BW}_{-3\mathrm{dB}}} = 14.4514
\end{gather}

\begin{figure}[H]\centering
    \includegraphics[width=\columnwidth]{NCE-03 Small-Signal LNA/assets/2025-11-27_21-12-17__NCE-03_freqResponse.pdf}
    \vspace*{-10mm}
    \caption{Frequency Response of The Resonant Amplifier (Signal Gain $A_s$ and Voltage Gain $A_v$ vs. Frequency $f_{in}$)}
    \label{fig: freq response}
\end{figure}

\begin{figure}[H]\centering
    \includegraphics[width=\columnwidth]{NCE-03 Small-Signal LNA/assets/2025-11-27_21-19-22__NCE-03_fittedCurve_.pdf}
    \vspace*{-10mm}
    \caption{Fitted Signal Gain Frequency Response $A_s(f)$ of The Resonant Amplifier}
    \label{fig: fitted freq response}
\end{figure}

从 Figure \ref{fig: freq response} 中我们观察到 voltage gain $A_v = \frac{v_{out}}{v_{in}}$ 在 9.6 MHz 和 10.6 MHz 附近出现了两个峰值,猜测是由于两级放大器的谐振频率设置不当 (过于远离) 导致的“双峰现象”。\textbf{随后多次尝试调整可变电容 C7/C12 (例如 Figure \ref{fig: adjusted freq response}),但都未能有效改善该现象。} 又注意到 9.6 MHz 出现峰值是因为输入信号幅度 $v_{in,amp}$ 异常降低,从 17 mVamp 降低到 7 mVamp 左右,导致计算得到的 voltage gain $A_v$ 异常升高,因此原来的“谐振频率过于远离”猜测并不成立,估计是放大器第一级本身的输入阻抗导致,具体原因有待进一步探究。

\begin{figure}[H]\centering
    \includegraphics[width=\columnwidth]{NCE-03 Small-Signal LNA/assets/2025-11-27_21-28-22__NCE-03_adjustedFreqResponse.pdf}
    \caption{Frequency Response of The Resonant Amplifier After Adjusting C7/C12}
    \label{fig: adjusted freq response}
\end{figure}


\section{思考题}

\subsection{小信号调谐放大器产生自激振荡的原因是什么?如何避免产生自激振荡?}

本实验所用的小信号谐振放大器由两级构成,整体增益较高且 phase shift 较大。虽然电路中没有直接的反馈路径,但由于电路布局和元器件的寄生参数等原因 (尤其晶体管的 parasitic base-collector capacitor $C_{\pi}$, 也常记为 $C_{BE}$) ,仍然可能产生寄生反馈,一旦满足了自激振荡的起振条件 $|A_v(f_{osc})| = 1\ @\ \angle A_v(f_{osc}) = -180^\circ $,就会导致放大器产生自激振荡,输出端口出现持续的正弦波信号。此外,如果放大器的谐振频率设置不当 (例如两级谐振频率完全相等),也可能导致频率响应曲线过于尖锐,增加了自激振荡的风险。

可通过在输入或输出端添加补偿网络 (例如 RC 网络) 来降低避免自激振荡的产生。此外,合理调整谐振频率,使两级谐振频率略有差异,也有助于降低自激振荡风险。

\subsection{从频谱角度计算信号电压幅度时,除积分法之外还有哪些方法?}

在本实验,我们采取的方法是:先计算未经窗函数处理的时域信号的频谱,再用积分法从频谱中计算信号功率,进而折算为电压幅度。这种方法的优点是简单直接,但缺点是频谱泄露现象会影响计算精度。

事实上,学界/业界最常用的方法是:在时域信号上先乘以一个窗函数 (window function),例如 Hanning 窗、Hamming 窗或 Blackman 窗等,然后再计算频谱。加入窗函数后,频谱泄露现象会大大减弱,从而提高信号幅度计算的准确性。但是注意引入窗函数后,频谱幅度和功率有所衰减,需要对傅里叶结果进行矫正 (correction),具体矫正系数取决于所用窗函数的类型。

\newpage

作为一个例子,不妨设置被采样的原始信号为:
\begin{gather}
v(t) = 
(1 + A_{n1}(t)) \cdot \cos(2 \pi f_0 t + \phi_{n1}(t)) + 
(0.1 + A_{n2}(t)) \cdot \cos(2 \pi f_1 t + \phi_{n2}(t)) + 
V_{n}(t)
\\ 
\mathrm{where\ } f_0 = 1 \ \mathrm{kHz},\ f_1 = 2 \ \mathrm{kHz} \mathrm{\ and\ } A_{n},\ \phi_{n},\ V_{n} \mathrm{\ denote\ amplitude,\ phase,\ voltage\ noise,\ respectively.}
\end{gather}


然后对比其在不同窗函数下的频谱结果,结果如 Figure \ref{fig: window function comparison} 所示 (Uniform, Hanning and Flattop)。图中左半部分为时域信号乘以窗函数后的结果,右半部分为矫正后的功率谱 (Power Spectrum)。可以看出,三种不同窗函数 (Unitorm, Hanning, Flattop) 都能显示出 1 kHz 和 2 kHz 两个信号分量,但频谱泄露程度有所不同,Hanning 和 Flattop 窗函数的频谱泄露明显小于 Uniform (no window) 的结果。


\begin{figure}[H]\centering
    \includegraphics[width=\columnwidth]{NCE-03 Small-Signal LNA/assets/2025-11-29_09-01-05__NCE-03_window.pdf}
    \caption{Comparison of Different Window Functions (Uniform, Hanning and Flattop) in Time-Domain and Frequency-Domain}
    \label{fig: window function comparison}
\end{figure}


从离散频谱结果计算信号频率/幅度/功率的理论详见下面这篇文献,我们这里直接给出关键计算公式:

\begin{graybox}
[1] Brandt, A., 2011. Noise and vibration analysis: signal analysis and experimental procedures, John Wiley \& Sons, doi: 10.1002/9780470978160.
\end{graybox}\noindent\vspace*{-10mm}
\begin{gather}
\mathrm{Extracted\ Frequency:\ \ } f_0 = \frac{\displaystyle \sum_{n = n_0 - L}^{n_0 + L} \left[P[n] \cdot (n \Delta f)\right] }{\displaystyle \sum_{n = n_0 - L}^{n_0 + L} P[n]} \\
\mathrm{Corrected\ Power:\ \ } P_{0,cor} = \mathrm{ECF}\times \sum_{n = n_0 - L}^{n_0 + L} P[n] \\
\mathrm{Corrected\ Amplitude:\ \ } A_{0,cor} = \sqrt{2 \times P_{0,cor}}
\end{gather}
where:
\begin{enumerate}
\item $P[n]$ is the single-sided power spectrum obtained from DFT;
\item $n_0$ is the index of the frequency bin closest to the target signal frequency $f_0$;
\item $\Delta f = f_s/N$ is the frequency resolution;
\item $L$ denotes the half width of the integration window;
\item ECF is the Energy Correction Factor for the used window function, which is exactly 1 for Uniform window (no window).
\end{enumerate}

三种窗函数的积分结果 (主信号频率/幅度/功率计算结果) 如 Figure \ref{fig: window function calculation comparison} 所示。图中可以看出,在当前信号和采样参数下,Hanning 窗函数收敛速度最快,计算结果最接近真实值,Flattop 窗函数次之,Uniform 窗函数 (无窗函数) 收敛最慢,且频率收敛结果偏离真实值较多。

\begin{figure}[H]\centering
    \includegraphics[width=\columnwidth]{NCE-03 Small-Signal LNA/assets/2025-11-29_09-21-14__NCE-03_window_cal.pdf}
    \caption{Comparison of Different Window Functions in Signal Frequency, Amplitude, and Power Calculation}
    \label{fig: window function calculation comparison}
\end{figure}

除上述方法外,还有一些其他方法可以从频谱角度计算信号功率/幅度,比如 “DFS-based method (直接在目标频率处对采样序列进行离散傅里叶级数展开)” 等,这里不多赘述。




\newpage
% 附录
\section*{附录 A\hspace*{20pt} 原始数据记录表}
\addcontentsline{toc}{section}{附录 A\hspace*{6pt} 原始数据记录表} 
\thispagestyle{fancy} 
\includegraphics[width=\columnwidth]{NCE-03 Small-Signal LNA/assets/data.jpg}

% 附录
\newpage
\vspace*{\fill}
\begin{center}\Huge{\bfseries 
    附录 B\hspace*{20pt} 实验预习报告
}\end{center}\addcontentsline{toc}{section}{附录 B\hspace*{6pt} 实验预习报告} 
\vspace*{\fill}
\thispagestyle{fancy} 
\includepdf[pages={-}]{NCE-03 Small-Signal LNA/preview/NCE-03 (preview report).pdf}


% 附录
\section*{附录C \hspace*{20pt} MATLAB Codes}
\addcontentsline{toc}{section}{附录 C \hspace*{6pt} MATLAB Codes} 
\thispagestyle{fancy} 
\lstinputlisting{d:/a_RemoteRepo/GH.MatlabCodes/本科课程代码/Non-Linear Circuits/NCE_03.m}





































\end{document}

% VScode 常用快捷键:

% F2:                       变量重命名
% Ctrl + Enter:             行中换行
% Alt + up/down:            上下移行
% 鼠标中键 + 移动:           快速多光标
% Shift + Alt + up/down:    上下复制
% Ctrl + left/right:        左右跳单词
% Ctrl + Backspace/Delete:  左右删单词    
% Shift + Delete:           删除此行
% Ctrl + J:                 打开 VScode 下栏(输出栏)
% Ctrl + B:                 打开 VScode 左栏(目录栏)
% Ctrl + `:                 打开 VScode 终端栏
% Ctrl + 0:                 定位文件
% Ctrl + Tab:               切换已打开的文件(切标签)
% Ctrl + Shift + P:         打开全局命令(设置)

% Latex 常用快捷键:

% Ctrl + Alt + J:           由代码定位到PDF


