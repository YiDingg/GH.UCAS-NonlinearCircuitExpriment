% 若编译失败,且生成 .synctex(busy) 辅助文件,可能有两个原因:
% 1. 需要插入的图片不存在:Ctrl + F 搜索 'figure' 将这些代码注释/删除掉即可
% 2. 路径/文件名含中文或空格:更改路径/文件名即可

% --------------------- 文章宏包及相关设置 --------------------- %
% >> ------------------ 文章宏包及相关设置 ------------------ << %
% 设定文章类型与编码格式
\documentclass[UTF8]{article}		
\input{../../.config/config_for_NonlinearCircuitExperiment.tex}



%%%%%%%%%%%%%%%%%%%%%%%%%%%%%%%%%%%%%%%%%%%%%%%%%%%%%%%%%%%%%%%%
% 仅需修改页眉、实验名称、实验日期
%%%%%%%%%%%%%%%%%%%%%%%%%%%%%%%%%%%%%%%%%%%%%%%%%%%%%%%%%%%%%%%%


%%%%%%%%%%%%%%%%%% 1. 修改页眉内容 %%%%%%%%%%%%%%%%%%
\rhead{Preview Report of NCE-04 PA (2025.11.27, 丁毅)}

% 开始编辑文章
\begin{document}
\begin{center}\large
    \vspace*{-0.8cm}
    \noindent{\huge\bfseries《\ \ 非\ \ 线\ \ 性\ \ 电\ \ 路\ \ 实\ \ 验\ \ \ 》\ \ 预\ \ 习\ \ 报\ \ 告 }
    \\\vspace{0.1cm}
    \noindent{
    {\bfseries 
%
%%%%%%%%%%%%%%%%%% 2. 修改实验名称 %%%%%%%%%%%%%%%%%%
    实验名称:\uline{\hspace{0.3cm} Large-Signal Power Amplifier  \hspace{0.3cm}}
%
    }\hspace{0.4cm}
    指导教师:\uline{\hspace{0.3cm}冯鹏\ \ fengpeng06@semi.ac.cn     \hspace{0.3cm}}
    }
    \\\vspace{0.1cm}
    \noindent
    {
    姓名:\uline{\,\,\,丁毅\,\,\,}\hspace{0.2cm}
    学号:\uline{\,\,\,{ 2023K8009908031}\,\,\,}\hspace{0.2cm}
    班级/专业:\uline{\,\,\,{2308/电子信息}\,\,\,}\hspace{0.2cm}
    分组序号:\uline{\,\,\,{2-06}\,\,\,}
    }
    \\\vspace{0.1cm}
    \noindent{
%
%%%%%%%%%%%%%%%%%% 3. 修改实验日期 %%%%%%%%%%%%%%%%%%
    实验日期:\uline{\,\,{2025.11.27}\,\,}\hspace{0.2cm}
%
    实验地点:\uline{\,\,\,西实验楼 (8 号楼) { 308}\,\,\,}\hspace{0.2cm}
    是否调课/补课:\uline{\hspace{0.1cm}否 \hspace{0.1cm}}\hspace{0.2cm}
    成绩:\uline{\hspace{0.6cm}}
    }
\end{center}
\vspace{-0.4cm}
\noindent\rule{\textwidth}{0.075em}   % 分割线
\vspace{-1.0cm}


% ------------------------ 文章信息区 ------------------------ %
% ------------------------ 文章信息区 ------------------------ %



%%%%%%%%%%%%%%%%%%%%%%%%%%%%%%%%%%%%%%%%%%%%%%%%%%%%%%%%%%%%%%%%%%%%%%%%%%%%%%%%%
%%%%%%%%%%%%%%%%%%%%%%%%%%%%%%%%% 下面是正文内容 %%%%%%%%%%%%%%%%%%%%%%%%%%%%%%%%%
%%%%%%%%%%%%%%%%%%%%%%%%%%%%%%%%%%%%%%%%%%%%%%%%%%%%%%%%%%%%%%%%%%%%%%%%%%%%%%%%%

\section{实验目的}



\begin{enumerate}
\item 掌握谐振功率放大器的工作原理;
\item 掌握谐振功率放大器的调谐特性和负载特性;
\item 掌握集电极电源电压及负载变化对放大器工作状态的影响。
\end{enumerate}




\section{实验仪器}

\begin{enumerate}
\item 谐振功率放大器实验板 (序列号 ……)
\item 示波器 RIGOL MSO2202A  (序列号 ……)
\item 信号发生器 GW INSTEK AFG-2225  (序列号 ……)
\item 万用表 LINIT- UT61A (序列号 ……)
\end{enumerate}



\section{实验原理}

\subsection{谐振功率放大器基本原理}

本实验中的使用的谐振功率放大器 (PA, Power Amplifier) 为 Class-C PA (丙类功放),采用谐振网络作为负载回路,且导通角 $\theta \in (0^\circ, 90^\circ)$。

功率放大器导通角余弦值 $\cos \theta = \frac{V_{TH} - V_{B}}{V_{in,amp}} < 180^\circ$ 时 (例如 Class-AB, Class-B 和 Class-C PA),尽管输出波形存在属于 “削波失真” (clipping distortion) 的 “截断” 现象,具有非常明显的非线性失真 (高次谐波分量较多),但由于负载为谐振网络,输出端口的电流和电压波形经过谐振网络滤波后,高次谐波被充分抑制,以此实现高效的大信号功率放大。

\begin{figure}[H]\centering
\begin{subfigure}[b]{0.4\columnwidth}\centering
    \includegraphics[height=170pt]{assets/operating principle.png}
    \caption{Schematic of basic class-C Power amplifier}
\end{subfigure}\hfill
\begin{subfigure}[b]{0.6\columnwidth}\centering
    \includegraphics[height=170pt]{assets/input voltage vs. output current.png}
    \caption{Input voltage vs. output current of class-C PA}
\end{subfigure}
\caption{Class-C Power Amplifier Operating Principle}
\end{figure}

~\\

理论分析已经在非线性电路课程中详细介绍过,以下仅给出主要公式:

\begin{gather}
\cos \theta = \frac{V_{TH} - V_{B}}{V_{in,amp}} ,\quad I_{\max} = g_m V_{in,amp} (1 - \cos \theta) 
\\ 
P_L := P_{out, base} = \frac{1}{2} I_{1,amp} V_{1,amp} = \frac{1}{2}  I_{1,amp}^2 R_L
\\
P_{Q} = I_Q V_{CC} = I_0 V_{CC} \Longrightarrow \eta_C = \frac{P_{L}}{P_{Q}} = \frac{1}{2} \frac{I_{1,amp} V_{1,amp} }{I_0 V_{CC}} = \frac{1}{2} \frac{\alpha_1 V_{1,amp}}{\alpha_0 V_{CC}}
\\ 
\alpha_0 = \frac{I_0}{I_{\max}} = \frac{\sin \theta - \theta \cos \theta}{\pi (1 - \cos \theta)},\quad \alpha_1 = \frac{I_{1,amp}}{I_{\max}} = \frac{\theta - \sin \theta \cos \theta}{\pi (1 - \cos \theta)}
\\ 
\alpha_n = \frac{2}{\pi}\cdot \frac{\sin n \theta  \cos \theta - n \cos n\theta \sin \theta}{n (n^2 - 1 ) (1 - \cos \theta)} ,\quad n = 2, 3, \cdots
\\ 
I_{n,amp} = \alpha_n I_{\max} = \alpha_n g_m V_{in,amp} (1 - \cos \theta)
\\ 
i(t) = \sum_{n=0}^{\infty} I_n \cos n \omega_0 t = I_0 + I_{1,amp} \cos (\omega_0 t) + I_{2,amp} \cos (2 \omega_0 t) + \cdots
\end{gather}

\begin{figure}[H]\centering
\begin{subfigure}[b]{0.5\columnwidth}\centering
    \includegraphics[height=140pt]{assets/PA current waveform.png}
    \caption{Current waveform of class-A/B/C power amplifier}
\end{subfigure}\hfill
\begin{subfigure}[b]{0.5\columnwidth}\centering
    \includegraphics[height=140pt]{assets/PA voltage waveform.png}
    \caption{Voltage waveform of class-A/B/C power amplifier}
\end{subfigure}
\caption{Operating Waveforms of Class-A/B/C Power Amplifiers}
\end{figure}

\subsection{谐振功率放大器的直流工作点}



\noindent 一个 Class-C PA 的工作状态由下面四个参数决定:
\begin{enumerate}
\item 集电极电源电压 $V_{CC}$,即讲义中的 $E_C$
\item 负载电阻 $R_L$,即讲义中的 $R_e$
\item 输入信号幅度 $V_{in,amp}$,即讲义中的 $U_{bm}$
\item 晶体管 Base 端偏置电压 $V_{B}$,即讲义中的 $U_{B}$
\end{enumerate} 

~

\noindent 由此衍生出几种特性曲线:
\begin{enumerate}
\item 负载特性:保持其它参数不变,改变负载电阻 $R_L$ 时,功率放大器的基频输出电流 $I_{1,amp}$、基频输出电压 $V_{1,amp}$、基频输出功率 $P_L$ 以及集电极效率 $\eta_C$ 随之变化的曲线;
\item 集电极调制特性 (电源调制特性):保持其它参数不变,改变电源电压 $V_{CC}$  (即集电极电压) 时,功率放大器的上述输出参数随之变化的曲线;
\item 输入调制特性:保持其它参数不变,改变输入信号幅度 $V_{in,amp}$ 时,功率放大器的上述输出参数随之变化的曲线;
\end{enumerate}



\subsection{实验所用功放电路分析}

\begin{figure}[H]\centering
    \includegraphics[width=\columnwidth]{assets/PA.png}
    \caption{Schematic of The Power Amplifier Used in This Experiment}
\end{figure}

该放大器有两级,第一级放大器在基极用可变电阻调节偏置,集电极负载是一个可调 LC 谐振回路,电源电压 (集电极电压) 为 12 V;第二级集电极负载仍为 LC 谐振回路,但电源电压 (集电极电压) 由 LDO 给出,低于 12 V。RW2 和 R8 为功放负载,也可以将输出接到天线负载。


\section{实验内容与步骤}





本次实验的注意内容为“放大器的频率特性及通频带的测量”。需注意:
\begin{enumerate}
\item 调整两级放大器的可调电容,使两级放大器的谐振频率分别略低于和略高于 10.7 MHz, 但不能完全相等或相差过大;完全相等会导致频率响应曲线过于尖锐,难以测量通频带,差距过大则会导致增益明显下降,甚至出现中间窄两边高的“双峰现象”;
\item 为电路注入小信号时,输入信号幅度 (amplitude) 应控制在 30 mV 以内,过大会导致放大器进入非线性工作区,输出波形严重失真;
\end{enumerate}

\subsection{功率放大器直流工作点调整}

\begin{enumerate}
\item 连接好电路,在实验箱主板上插上高频谐振功率放大器实验电路模块,接通实验箱电源并打开开关,指示灯点亮。由信号发生器提供 50 mVpp sinusoidal wave @ 10 MHz 给到功放的 IN1 输入端。
\item 在 OUT 端用示波器观测到放大后的输入信号,调整电位器 RW1/RW2,微调可调电容 CV,在 OUT 端用示波器观察输出信号状态。
\end{enumerate}



\subsection{负载调制特性测量}

\begin{enumerate}
\item 调整 RW3 使第二级功放的电源电压为最大值 (测量 TP5);
\item 将负载接到电阻端,保持其它参数不变,改变负载 RL,观察电压、电流波形的变化情况;
\item 具体而言,改变 $R_L$ 的同时,测量输出电压 $V_{1,amp}$、输出电流 $I_{1,amp}$ 和直流偏置电流 $I_0$,以此计算输出功率 $P_L$ 和集电极效率 $\eta_C$ 等其它参数。
\end{enumerate}

~

可先测量出 R8 的阻值,每次改变负载后断电测量 RW2 两端阻值,这样得到的是 $(R_8 \parallel RW_2)$,可通过数学运算得到负载 $R_L$ 的值,省去了频繁拔插跳线 J2 的麻烦步骤。

~

数据记录表格如下:

\begin{table}[H]\centering
    %\renewcommand{\arraystretch}{1.5} % 调整行间距
    %\setlength{\tabcolsep}{1mm} % 调整列间距
    \caption{Load Modulation Characteristic Data}
    \label{Load Modulation Characteristic Data}
\begin{tabular}{lcccccccccccccc}\toprule
    (断电后测量) $R_L$ (Ohm)   & & &  &  &  &  &  &  &  &  &  &  &  &  \\
    \midrule
    (AC of TP4) $V_{1,pp}$ (V) &  &  &  &  &  &  &  &  &  &  &  &  &  &  \\
    (AC of TP3) $I_{1,pp} R_7$ (V)    &  &  &  &  &  &  &  &  &  &  &  &  &  &  \\
    (DC of TP3) $I_0R_7$ (V)  &  &  &  &  &  &  &  &  &  &  &  &  &  &  \\
    \bottomrule
\end{tabular}
\end{table}



\textbf{注意:为避免板子烧坏,上电前将 J2 连接到电阻端,且负载电阻总阻值不能小于 50Ohm (不能空载),功放输出电压峰峰值不可超过 1.5V。}



\subsection{集电极调制特性测量}

\begin{enumerate}
\item 保持偏置电压 $V_B$ 不变,\textbf{负载接天线并且将天线展开将天线展开},不断调整 RW3 以改变第二级功放的 $V_{CC}\ (E_C)$,在 TP3 测量 $V_{CC}$ 对电压和电流波形的影响。
\item 具体而言,改变 $V_{CC}$ 的同时,需要测量输出电压 $V_{1,amp}$、输出电流 $I_{1,amp}$ 和直流偏置电流 $I_0$,以此计算输出功率 $P_L$ 和集电极效率 $\eta_C$ 等参数。
\end{enumerate}


数据记录表格如下:

\begin{table}[H]\centering
    %\renewcommand{\arraystretch}{1.5} % 调整行间距
    %\setlength{\tabcolsep}{1mm} % 调整列间距
    \caption{Collecter Modulation Characteristic Data}
    \label{Collecter Modulation Characteristic Data}
\begin{tabular}{lcccccccccccccc}\toprule
    (DC of TP5) $V_{CC}$ (V)   & & &  &  &  &  &  &  &  &  &  &  &  &  \\
    \midrule
    (AC of TP4) $V_{1,pp}$ (V) &  &  &  &  &  &  &  &  &  &  &  &  &  &  \\
    (AC of TP3) $I_{1,pp} R_7$ (V)    &  &  &  &  &  &  &  &  &  &  &  &  &  &  \\
    (DC of TP3) $I_0R_7$ (V)  &  &  &  &  &  &  &  &  &  &  &  &  &  &  \\
    \bottomrule
\end{tabular}
\end{table}


\section{思考题}

\subsection{根据实验电路,分析可能会造成实验电路损坏的原因,应该采取哪些预防措施?}

\noindent 可能的原因:
\begin{enumerate}
\item 输入信号幅值过大或放大倍率过大,导致输出功率过大,板子烧毁;
\item 上电未接负载,导致输出电流过大,板子烧毁;
\item 断开条线 J2 前未断电,产生较强冲激电流,烧毁板子;
\item 板子以过大功率工作较长时间,导致负载端过热熔毁。
\end{enumerate}

\noindent 预防措施:
\begin{enumerate}
\item 合理控制放大倍数和输入电压幅值,控制输出功率;
\item 规范操作,上电前先接入负载,调节电路时断电;
\item 控制电路板工作时间,避免长时间工作导致过热。
\end{enumerate}









































\end{document}

% VScode 常用快捷键:

% F2:                       变量重命名
% Ctrl + Enter:             行中换行
% Alt + up/down:            上下移行
% 鼠标中键 + 移动:           快速多光标
% Shift + Alt + up/down:    上下复制
% Ctrl + left/right:        左右跳单词
% Ctrl + Backspace/Delete:  左右删单词    
% Shift + Delete:           删除此行
% Ctrl + J:                 打开 VScode 下栏(输出栏)
% Ctrl + B:                 打开 VScode 左栏(目录栏)
% Ctrl + `:                 打开 VScode 终端栏
% Ctrl + 0:                 定位文件
% Ctrl + Tab:               切换已打开的文件(切标签)
% Ctrl + Shift + P:         打开全局命令(设置)

% Latex 常用快捷键:

% Ctrl + Alt + J:           由代码定位到PDF


