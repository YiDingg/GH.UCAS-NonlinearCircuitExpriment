% 若编译失败,且生成 .synctex(busy) 辅助文件,可能有两个原因:
% 1. 需要插入的图片不存在:Ctrl + F 搜索 'figure' 将这些代码注释/删除掉即可
% 2. 路径/文件名含中文或空格:更改路径/文件名即可

% --------------------- 文章宏包及相关设置 --------------------- %
% >> ------------------ 文章宏包及相关设置 ------------------ << %
% 设定文章类型与编码格式
\documentclass[UTF8]{article}		
\input{../../.config/config_for_NonlinearCircuitExperiment.tex}



%%%%%%%%%%%%%%%%%%%%%%%%%%%%%%%%%%%%%%%%%%%%%%%%%%%%%%%%%%%%%%%%
% 仅需修改页眉、实验名称、实验日期
%%%%%%%%%%%%%%%%%%%%%%%%%%%%%%%%%%%%%%%%%%%%%%%%%%%%%%%%%%%%%%%%


%%%%%%%%%%%%%%%%%% 1. 修改页眉内容 %%%%%%%%%%%%%%%%%%
\rhead{Preview Report of NCE-06 Mixer (2025.12.04, 丁毅)}

% 开始编辑文章
\begin{document}
\begin{center}\large
    \vspace*{-0.8cm}
    \noindent{\huge\bfseries《\ \ 非\ \ 线\ \ 性\ \ 电\ \ 路\ \ 实\ \ 验\ \ \ 》\ \ 预\ \ 习\ \ 报\ \ 告 }
    \\\vspace{0.1cm}
    \noindent{
    {\bfseries 
%
%%%%%%%%%%%%%%%%%% 2. 修改实验名称 %%%%%%%%%%%%%%%%%%
    实验名称:\uline{\hspace{0.8cm} Multiplier-Based Mixer \hspace{0.8cm}}
%
    }\hspace{0.4cm}
    指导教师:\uline{\hspace{0.5cm}冯鹏\ \ fengpeng06@semi.ac.cn     \hspace{0.5cm}}
    }
    \\\vspace{0.1cm}
    \noindent
    {
    姓名:\uline{\,\,\,丁毅\,\,\,}\hspace{0.2cm}
    学号:\uline{\,\,\,{ 2023K8009908031}\,\,\,}\hspace{0.2cm}
    班级/专业:\uline{\,\,\,{2308/电子信息}\,\,\,}\hspace{0.2cm}
    分组序号:\uline{\,\,\,{2-06}\,\,\,}
    }
    \\\vspace{0.1cm}
    \noindent{
%
%%%%%%%%%%%%%%%%%% 3. 修改实验日期 %%%%%%%%%%%%%%%%%%
    实验日期:\uline{\,\,{2025.12.04}\,\,}\hspace{0.2cm}
%
    实验地点:\uline{\,\,\,西实验楼 (8 号楼) { 308}\,\,\,}\hspace{0.2cm}
    是否调课/补课:\uline{\hspace{0.1cm}否 \hspace{0.1cm}}\hspace{0.2cm}
    成绩:\uline{\hspace{0.6cm}}
    }
\end{center}
\vspace{-0.4cm}
\noindent\rule{\textwidth}{0.075em}   % 分割线
\vspace{-1.0cm}


% ------------------------ 文章信息区 ------------------------ %
% ------------------------ 文章信息区 ------------------------ %



%%%%%%%%%%%%%%%%%%%%%%%%%%%%%%%%%%%%%%%%%%%%%%%%%%%%%%%%%%%%%%%%%%%%%%%%%%%%%%%%%
%%%%%%%%%%%%%%%%%%%%%%%%%%%%%%%%% 下面是正文内容 %%%%%%%%%%%%%%%%%%%%%%%%%%%%%%%%%
%%%%%%%%%%%%%%%%%%%%%%%%%%%%%%%%%%%%%%%%%%%%%%%%%%%%%%%%%%%%%%%%%%%%%%%%%%%%%%%%%

\section{实验目的}

\begin{enumerate}
\item 进一步学习变频电路的相关理论;
\item 掌握乘法混频电路的工作原理和调试方法。
\end{enumerate}



\section{实验仪器}

\begin{enumerate}
\item 高频实验箱 - 集成乘法调幅/混频实验板 (……)
\item 示波器 RIGOL MSO2202A  (……)
\item 信号发生器 GW INSTEK AFG-2225  (……)
\item 万用表 LINIT- UT61A (……)
\end{enumerate}



\section{实验原理}

\subsection{Basic Principles of Mixer}

在通信系统中,经常需要将信号自某一频率变换为另一频率,例如把
一个已调的高频信号变成另一个较低频率的同类已调信号 (Downconversion, 下变频),又或者把一个已调的低频信号变成另一个较高频率的同类已调信号 (Upconversion, 上变频)。完成这种频率变换的电路称为变频器或混频器,它是无线通信系统中的重要组成部分。

混频器 (Mixer) 通常由非线性器件作为核心 (二极管、晶体管、模拟乘法器模块),配合其它有源模块和无源器件组成,这些器件会将输入信号注入到非线性器件中,由其完成混频操作。近年混频器方面的技术前沿表明,混频器的带宽主要受到无源器件限制,而非受到二极管或晶体管的限制,因为后者的带宽一般能很好地满足要求。因此,设计高性能混频器的关键在于合理设计无源器件网络,以实现所需的频率响应特性。

作为射频系统中的关键组件,混频器的主要参数有:

\begin{enumerate}
\item 功率增益 (Power Gain):$G_{p} = \frac{P_{out}}{P_{in}}$,表示输出信号功率与输入信号功率之比;
\item 噪声系数 (Noise Figure):$NF = \frac{SNR_{in}}{SNR_{out}}$,表示输入信号与输出信号的信噪比之比;
\item 混频失真与干扰:混频器的失真包括频率失真等非线性失真,此外由于器件的非线
性还存在着组合频率干扰,这些干扰往往伴随有用信号而存在,严重影响混频器正常工
作,因此需关注减小混频失真与干扰;
\item 选择性:指混频器选出有用输出信号而滤除其他干扰信号的能力,选择性越好输出的频谱纯度越高,往往取决于输出端的带通滤波器的性能。
\end{enumerate}

\subsection{实验电路简要分析}

本次实验采用的 Multiplier-Based Mixer (乘法器混频器) 如 Figure \ref{fig: multiplier_based_mixer} 所示,用于将高频输入信号 $f_{in}$ 下变频 (Downconversion) 为中频输出信号 $f_{out}$。

该电路整体上也分为两级,对混频功能而言,第一级为由模拟乘法器构成的混频电
路,配合外围的电容电阻来控制输入信号的幅度;频率为 455 kHz 的输出经电容耦合接入作为输出隔离的第二级 (Common Collector)。


\begin{figure}[H]\centering
    \includegraphics[width=\columnwidth]{assets/circuit.png}
    \caption{Multiplier-Based Mixer Schematic}
    \label{fig: multiplier_based_mixer}
\end{figure}




实验时,将跳线 J1, J2, J3 的二号端口连接到一号端口 (也即“混频”功能,连接到三号端口对应“调幅”功能),并在 IN1 (TP1) 和 IN3 (TP3) 分别输入本振 $f_{in} = 10.245 \mathrm{MHz}$ 和载波 $f_c = 10.700 \ \mathrm{MHz}$,此时混频器输出中频信号 $f_{out} = |f_{in} - f_c| = 455 \mathrm{kHz}$。

\section{实验内容与步骤}






\begin{enumerate}
\item 在实验箱上插上集成乘法器混频电路实验模块和正弦波振荡电路实验模块,接
通实验箱电源。
\item 在 IN1 端接入由信号发生器产生的 10.245 MHz 本振信号;然后利用 NCE-05 实验中的正弦波振荡电路输出 10.7 MHz 载波信号,接入混频器模块 IN3 (TP3) 端;
\item 调节 IN1 输入信号幅度和电阻 W1,用示波器采样输出波形,将数据导出到电脑用 MATLAB 进行傅里叶分析,直至得到 455 kHz 输出信号;
\item 调节可变电阻 W2,继续用 MATLAB 对输出信号进行傅里叶分析,直至得到幅度合适、信噪比最好、失真度最小的 455 kHz 输出信号;
\end{enumerate}



\section{思考题}

\subsection{改变信号发生器提供的 IN1 输入频率时,输出波形和频率作何变化,为什么?}




































\end{document}

% VScode 常用快捷键:

% F2:                       变量重命名
% Ctrl + Enter:             行中换行
% Alt + up/down:            上下移行
% 鼠标中键 + 移动:           快速多光标
% Shift + Alt + up/down:    上下复制
% Ctrl + left/right:        左右跳单词
% Ctrl + Backspace/Delete:  左右删单词    
% Shift + Delete:           删除此行
% Ctrl + J:                 打开 VScode 下栏(输出栏)
% Ctrl + B:                 打开 VScode 左栏(目录栏)
% Ctrl + `:                 打开 VScode 终端栏
% Ctrl + 0:                 定位文件
% Ctrl + Tab:               切换已打开的文件(切标签)
% Ctrl + Shift + P:         打开全局命令(设置)

% Latex 常用快捷键:

% Ctrl + Alt + J:           由代码定位到PDF


